% made with beamer theme arguelles v2.5.0
% theme  author: Michele Piazzai
% theme github https://piazzai.github.io
% theme  license: MIT

\documentclass[compress,12pt]{beamer}
\usepackage{framed}

\usetheme{Arguellespell}
\usepackage{minted}
\setminted{style=one-dark}
\usepackage{xcolor}
\usepackage[svgnames]{xcolor}
\usepackage{hyperref}
\hypersetup{
  colorlinks=true,
  linkcolor=CornflowerBlue,
  urlcolor=CornflowerBlue,
  citecolor=CornflowerBlue,
  pdfborderstyle={/S/U/W 1}
}
\usepackage{csquotes}
\definecolor{OneDarkBG}{HTML}{3B414D}

\title{Getting To Know My Enemies}
\subtitle{Time Classes and Time Zones in Rails}
\event{B'more on Rails}
\date{December 11, 2025}
\author{Rosanna Speller}

\github{r-spell}
\gitlab{r-spell}
\linkedin{rlspeller}

\begin{document}

\begin{frame}[plain]
\titlepage

\textcolor{white}{These slides are at} \url{http://bit.ly/44Y6iYf}
\end{frame}



\Section{}
\begin{frame}
 \frametitle{About me}
  \begin{itemize}
   \item Learned to code in Python \& Java
   \item 8 years working in Rails (with some JS)
   \begin{itemize}
    \item 6 years as the "Rails Expert" of my team
    \item Work at a smallish tech non-profit (launchcode.org)
    \end{itemize}
    \item This is my first tech talk!
  \end{itemize}
\end{frame}

\begin{frame}
 \frametitle{My enemies}
  \begin{itemize}
    \item Timezones
    \item Daylight savings time
    \item The letter S
  \end{itemize}
\end{frame}

\begin{frame}
 \frametitle{Versions}
  \begin{itemize}
    \item Rails 8.1
    \item Ruby 3.4
  \end{itemize}
\end{frame}

\begin{frame}
 \frametitle{Examples}
  Roughly based on Rails 8.1 "Getting Started" App\footnote{ \href{https://guides.rubyonrails.org/v8.1/getting_started.html}{Rails Guide Getting Started}}
  \begin{itemize}
    \item An online store
    \item Products and subscribers
  \end{itemize}

  
\end{frame}

\begin{frame}[containsverbatim]
\frametitle{Application Config}
\framesubtitle{boiler plate info}

In \mintinline{ruby}{config/application.rb}
\begin{framed}
\begin{minted}[fontsize=\scriptsize,style=sas,breaklines]{ruby}

    # These settings can be overridden in specific environments using the files
    # in config/environments, which are processed later.
    #
    # config.time_zone = "Central Time (US & Canada)"
    # config.eager_load_paths << Rails.root.join("extras")
    
\end{minted}
\end{framed}
\end{frame}

\begin{frame}[containsverbatim]
\frametitle{Application Config}
\framesubtitle{Set CA Time Zone}
In \mintinline{ruby}{config/application.rb}
\begin{framed}
\begin{minted}[fontsize=\scriptsize, style=sas,breaklines]{ruby}

    # These settings can be overridden in specific environments using the files
    # in config/environments, which are processed later.
    #
    config.time_zone = "Pacific Time (US & Canada)"
    # config.eager_load_paths << Rails.root.join("extras")
    
\end{minted}
\end{framed}
\end{frame}

\begin{frame}[containsverbatim]
\frametitle{Application Config}
\framesubtitle{Set CA Time Zone}
\centering
\includegraphics[width=0.75\textwidth]{firegirl.jpg}

\footnotesize Image from https://imgflip.com/memegenerator/7923908/fire-girl
\end{frame}


\begin{frame}[containsverbatim]
\frametitle{date, datetime, time}
\framesubtitle{in the database migration}
\begin{minted}[fontsize=\scriptsize, bgcolor=OneDarkBG, bgcolorpadding=1em]{ruby}

class AddColumnsToProducts < ActiveRecord::Migration[8.1]
  def change
    add_column :products, :launch_date, :date
    add_column :products, :launch_datetime, :datetime
    add_column :products, :launch_time, :time
  end
end

\end{minted}
\end{frame}

\begin{frame}[containsverbatim]
\frametitle{date, datetime, time}
\framesubtitle{in the database schema}
\begin{minted}[fontsize=\scriptsize,bgcolor=OneDarkBG, bgcolorpadding=1em]{ruby}

# db/schema.rb
# ...
  create_table "products", force: :cascade do |t|
    t.datetime "created_at", null: false
    t.integer "inventory_count"
    t.date "launch_date"
    t.datetime "launch_datetime"
    t.time "launch_time"
    t.string "name"
    t.datetime "updated_at", null: false
  end

\end{minted}
\end{frame}

\begin{frame}[containsverbatim]
\frametitle{date, datetime, time}
\framesubtitle{in the form helpers}
\begin{minted}[fontsize=\scriptsize,bgcolor=OneDarkBG, bgcolorpadding=1em]{erb}

<div>
  <%= form.label :launch_date, style: "display: block" %>
  <%= form.date_field :launch_date %>
</div>

<div>
  <%= form.label :launch_datetime, style: "display: block" %>
  <%= form.datetime_field :launch_datetime %>
</div>

<div>
  <%= form.label :launch_time, style: "display: block" %>
  <%= form.time_field :launch_time %>
</div>

\end{minted}
\end{frame}


\begin{frame}[containsverbatim]
\frametitle{date, datetime, time}
\framesubtitle{in the form helpers (part 2)}
\includegraphics[width=0.75\textwidth]{formhelpers.png}
\end{frame}

\begin{frame}[containsverbatim]
\frametitle{date, datetime, time}
\framesubtitle{in the form helpers (part 3)}
\includegraphics[width=0.75\textwidth]{filledformhelpers.png}
\end{frame}

\begin{frame}[containsverbatim]
\frametitle{date, datetime, time}
\framesubtitle{what about the parameters?}

\normalsize Using the Rails \mintinline{ruby}{debug} gem:\footnote{\href{https://guides.rubyonrails.org/v8.1/debugging_rails_applications.html\#debugging-with-the-debug-gem}{Rails Guide: Debug Gem}}


\begin{minted}[fontsize=\tiny,bgcolor=OneDarkBG, bgcolorpadding=1em,breaklines]{ruby}

[45, 53] in ~/projects/timewarp/app/controllers/products_controller.rb
    45| 
    46|     def product_params
    47|       x = params.expect(product:
    48|         [ :name,:description, :featured_image, :inventory_count, :launch_date, :launch_datetime, :launch_time ]
    49|       )
=>  50|       debugger
    51|       x
    52|     end
    53| end
\end{minted}

\end{frame}


\begin{frame}[containsverbatim]
\frametitle{date, datetime, time}
\framesubtitle{what about the parameters?}


See the params \mintinline{ruby}{x} using the debugger:
\begin{framed}
 \begin{minted}[fontsize=\tiny, style=sas,breaklines]{ruby}
=>  50|       debugger
    51|       x
    52|     end
    53| end
=>#0	ProductsController#product_params at ~/projects/timewarp/app/controllers/products_controller.rb:50
  #1	ProductsController#create at ~/projects/timewarp/app/controllers/products_controller.rb:17
  # and 81 frames (use `bt' command for all frames)
(rdbg) x
#<ActionController::Parameters {"name" => "Example Product", "description" => "<div>Blah</div>", "inventory_count" => "2", "launch_date" => "2025-12-24", "launch_datetime" => "2025-12-24T12:34", "launch_time" => "12:34"} permitted: true>
(rdbg) 
\end{minted}
\end{framed}
\end{frame}


\begin{frame}[containsverbatim]
\frametitle{date, datetime, time}
\framesubtitle{what about the parameters?}


Use command  \mintinline{ruby}{continue} to complete the action:
\begin{framed}
 \begin{minted}[fontsize=\tiny, style=sas,breaklines]{ruby}
=>  50|       debugger
    51|       x
    52|     end
    53| end
=>#0	ProductsController#product_params at ~/projects/timewarp/app/controllers/products_controller.rb:50
  #1	ProductsController#create at ~/projects/timewarp/app/controllers/products_controller.rb:17
  # and 81 frames (use `bt' command for all frames)
(rdbg) x
#<ActionController::Parameters {"name" => "Example Product", "description" => "<div>Blah</div>", "inventory_count" => "2", "launch_date" => "2025-12-24", "launch_datetime" => "2025-12-24T12:34", "launch_time" => "12:34"} permitted: true>
(rdbg) continue    # command
  TRANSACTION (0.0ms)  BEGIN immediate TRANSACTION /*action='create',application='Timewarp',controller='products'*/

\end{minted}
\end{framed}
\end{frame}



\begin{frame}[containsverbatim]
\frametitle{date, datetime, time}
\framesubtitle{in the Rails Console}
\begin{minted}[fontsize=\scriptsize,bgcolor=OneDarkBG, bgcolorpadding=1em,breaklines]{ruby}

timewarp(dev):001> Product.last
  Product Load (0.3ms)  SELECT "products".* FROM "products" ORDER BY "products"."id" DESC LIMIT 1 /*application='Timewarp'*/
=> 
#<Product:0x0000000129b8fd60
 id: 10,
 created_at: "2025-12-08 10:09:53.662802000 -0800",
 name: "Example Product",
 updated_at: "2025-12-08 10:09:53.668152000 -0800",
 inventory_count: 2,
 launch_date: "2025-12-24",
 launch_datetime: "2025-12-24 12:34:00.000000000 -0800",
 launch_time: "2000-01-01 12:34:00.000000000 -0800">

\end{minted}
\end{frame}

\begin{frame}
  \frametitle{Classes}
  \framesubtitle{Date, DateTime, Time}

We looked at the form inputs and column types:
\begin{itemize}
     \item \mintinline{ruby}{date}
     \item \mintinline{ruby}{datetime} 
     \item  \mintinline{ruby}{time}
\end{itemize}

There's also Classes:
\begin{itemize}
     \item \mintinline{ruby}{Date}
     \item \mintinline{ruby}{DateTime} 
     \item  \mintinline{ruby}{Time}
\end{itemize}
\end{frame}

\begin{frame}
\frametitle{Date, DateTime, Time}
\framesubtitle{In Ruby and Rails}
  
   \mintinline{ruby}{Date}, \mintinline{ruby}{DateTime} and \mintinline{ruby}{Time} 
    are the each types of Ruby objects.\footnote{Ruby Docs for \href{https://docs.ruby-lang.org/en/3.2/Date.html}{\mintinline{ruby}{Date}}, \href{https://docs.ruby-lang.org/en/3.2/DateTime.html}{\mintinline{ruby}{DateTime}} and \href{https://docs.ruby-lang.org/en/3.2/Time.html}{\mintinline{ruby}{Time}}}
    
    Rails has \mintinline{ruby}{ActiveSupport} core extensions for these 3 classes.\footnote{Rails Docs for
    \href{https://api.rubyonrails.org/v7.0/classes/Date.html}{\mintinline{ruby}{Date}},  \href{https://api.rubyonrails.org/v7.0/classes/DateTime.html}{\mintinline{ruby}{DateTime}} and \href{https://api.rubyonrails.org/v7.0/classes/Time.html}{\mintinline{ruby}{Time}}
}
 \end{frame}

\begin{frame}[standout]
  \centering\large
Do instances of the \mintinline{ruby}{Date}, \mintinline{ruby}{DateTime} and \mintinline{ruby}{Time} classes have values similar to values we saw for \mintinline{ruby}{date}, \mintinline{ruby}{datetime} and \mintinline{ruby}{time} column types, and form helpers?
\end{frame}

\begin{frame}[containsverbatim]
\frametitle{Date, DateTime, Time}
\framesubtitle{in the console}

\begin{minted}[fontsize=\scriptsize,bgcolor=OneDarkBG, bgcolorpadding=1em]{ruby}

timewarp(dev):013> date = Date.today
=> Sun, 07 Dec 2025
timewarp(dev):014> date_time = DateTime.now
=> Sun, 07 Dec 2025 17:03:13 -0500
timewarp(dev):015> time = Time.now
=> 2025-12-07 17:03:17.282426 -0500

\end{minted}

(I made the examples on Sunday)

\end{frame}

\begin{frame}[containsverbatim]
\frametitle{Date, DateTime, Time}
\framesubtitle{in the console (cont.)}

\begin{minted}[fontsize=\scriptsize,bgcolor=OneDarkBG, bgcolorpadding=1em]{ruby}

timewarp(dev):013> date = Date.today
=> Sun, 07 Dec 2025
timewarp(dev):014> date_time = DateTime.now
=> Sun, 07 Dec 2025 17:03:13 -0500
timewarp(dev):015> time = Time.now
=> 2025-12-07 17:03:17.282426 -0500
timewarp(dev):016> date.acts_like?(:date)
=> true
timewarp(dev):017> date.acts_like?(:time)
=> false
timewarp(dev):018> date_time.acts_like?(:date)
=> true
timewarp(dev):019> date_time.acts_like?(:time)
=> true
timewarp(dev):020> time.acts_like?(:time)
=> true
 
 

\end{minted}
\end{frame}

\begin{frame}[containsverbatim]
\frametitle{Date, DateTime, Time}
\framesubtitle{in the console (cont.)}

\begin{minted}[fontsize=\scriptsize,bgcolor=OneDarkBG, bgcolorpadding=1em]{ruby}

timewarp(dev):013> date = Date.today
=> Sun, 07 Dec 2025
timewarp(dev):014> date_time = DateTime.now
=> Sun, 07 Dec 2025 17:03:13 -0500
timewarp(dev):015> time = Time.now
=> 2025-12-07 17:03:17.282426 -0500
timewarp(dev):016> date.acts_like?(:date)
=> true
timewarp(dev):017> date.acts_like?(:time)
=> false
timewarp(dev):018> date_time.acts_like?(:date)
=> true
timewarp(dev):019> date_time.acts_like?(:time)
=> true
timewarp(dev):020> time.acts_like?(:time)
=> true
timewarp(dev):021> time.acts_like?(:date)
=> false

\end{minted}
\end{frame}

\begin{frame}[standout]
  \centering\large
  \mintinline{ruby}{DateTime} is acts like a date and a time, so that must be the best one if I want to give a date and a time, right?
    
\end{frame}

\begin{frame}[standout]
  
\centering\Huge NO!

  \large 
     \mintinline{ruby}{DateTime} is actually \textbf{deprecated!}
  
  \end{frame}
  
  
\begin{frame}
\frametitle{DateTime vs.Time}
\framesubtitle{Prefer Time to DateTime}
  
  At least as far back as Ruby 3.0, Ruby says:
  
  \begin{displayquote}
   \mintinline{ruby}{DateTime} class is considered deprecated. Use  \mintinline{ruby}{Time} class.
    \footnote{ \href{https://docs.ruby-lang.org/en/3.0/DateTime.html}{Ruby Docs }}
   \end{displayquote}
   
  \end{frame}
  
  
   \begin{frame}
\frametitle{DateTime vs. Time}
\framesubtitle{Except when you prefer DateTime}

  
Current (3.4) Ruby still say that \mintinline{ruby}{DateTime} can still be preferable to  \mintinline{ruby}{Time} for \textbf{historical} dates.

\vfill

   See Ruby Docs for some interesting examples!\footnote{ \href{https://docs.ruby-lang.org/en/3.4/DateTime.html}{Read the Ruby Docs!}}
\end{frame}
  

\begin{frame}
\frametitle{DateTime vs. Time}
\framesubtitle{Why different?}

\mintinline{ruby}{DateTime}
\begin{itemize}
  \item subclass of \mintinline{ruby}{Date}
  \item can use old calendars (eg. Gregorian vs. Julian)
  \item doesn't deal with leap seconds or day light savings\footnote{ \href{https://docs.ruby-lang.org/en/3.4/DateTime.html}{Ruby \mintinline{ruby}{DateTime} Docs}}
\end{itemize}


  
\mintinline{ruby}{Time}
\begin{itemize}
  \item based internally on nanoseconds since "Unix Epoch"
  \item mostly deals with leap seconds
  \item deals with day light savings\footnote{ \href{https://docs.ruby-lang.org/en/3.4/Time.html}{Ruby \mintinline{ruby}{Time} Docs}}
\end{itemize}
 
\end{frame}
  
 \begin{frame}[containsverbatim]
\frametitle{date, datetime, time}
\framesubtitle{back in the console}
Let's look at the classes of the values we have for  \mintinline{ruby}{launch_date},  \mintinline{ruby}{launch_time} and \mintinline{ruby}{launch_datetime}.
\begin{minted}[fontsize=\scriptsize,bgcolor=OneDarkBG, bgcolorpadding=1em]{ruby}

timewarp(dev):011> product = Product.last
timewarp(dev):012> product.launch_date.class
=> Date
timewarp(dev):013> product.launch_time.class
=> ActiveSupport::TimeWithZone

\end{minted}
\end{frame}

  \begin{frame}[containsverbatim]
\frametitle{date, datetime, time}
\framesubtitle{back in the console}
Let's look at the classes of the values we have for  \mintinline{ruby}{launch_date},  \mintinline{ruby}{launch_time} and \mintinline{ruby}{launch_datetime}.
\begin{minted}[fontsize=\scriptsize,bgcolor=OneDarkBG, bgcolorpadding=1em]{ruby}

timewarp(dev):011> product = Product.last
timewarp(dev):012> product.launch_date.class
=> Date
timewarp(dev):013> product.launch_time.class
=> ActiveSupport::TimeWithZone
timewarp(dev):014> product.launch_datetime.class
=> ActiveSupport::TimeWithZone

\end{minted}
\end{frame}


  \begin{frame}[containsverbatim]
\frametitle{TimeWithZone}
\framesubtitle{What's the zone?}
 \mintinline{ruby}{launch_time} and \mintinline{ruby}{launch_datetime} are Rails  \mintinline{ruby}{ActiveSupport::TimeWithZone}s.
 
 Let's look at the  \mintinline{ruby}{zone}s: 
\begin{minted}[fontsize=\scriptsize,bgcolor=OneDarkBG, bgcolorpadding=1em]{ruby}


timewarp(dev):013> product.launch_time.class
=> ActiveSupport::TimeWithZone
timewarp(dev):014> product.launch_datetime.class
=> ActiveSupport::TimeWithZone
timewarp(dev):015> product.launch_datetime.zone
=> "PST"
timewarp(dev):016> product.launch_time.zone
=> "PST"

\end{minted}
\end{frame}

\begin{frame}[containsverbatim]
\frametitle{Time.now and Time.current}
\framesubtitle{in the console}

\begin{minted}[fontsize=\scriptsize,bgcolor=OneDarkBG, bgcolorpadding=1em]{ruby}
timewarp(dev):017> now = Time.now
=> 2025-12-07 20:18:32.345682 -0500
timewarp(dev):018> now.class
=> Time
timewarp(dev):019> current = Time.current
=> 2025-12-07 17:18:50.066319000 PST -08:00
timewarp(dev):020> current.class
=> ActiveSupport::TimeWithZone
\end{minted}

\end{frame}

\begin{frame}[containsverbatim]
\frametitle{Time.now and Time.current}
\framesubtitle{in the console cont.}

\begin{minted}[fontsize=\scriptsize,bgcolor=OneDarkBG, bgcolorpadding=1em]{ruby}
timewarp(dev):017> now = Time.now
=> 2025-12-07 20:18:32.345682 -0500
timewarp(dev):018> now.class
=> Time
timewarp(dev):019> current = Time.current
=> 2025-12-07 17:18:50.066319000 PST -08:00
timewarp(dev):020> current.class
=> ActiveSupport::TimeWithZone
timewarp(dev):021> current.zone
=> "PST"


\end{minted}
\end{frame}

\begin{frame}[containsverbatim]
\frametitle{Time.now and Time.current}
\framesubtitle{in the console cont.}

\begin{minted}[fontsize=\scriptsize,bgcolor=OneDarkBG, bgcolorpadding=1em]{ruby}
timewarp(dev):017> now = Time.now
=> 2025-12-07 20:18:32.345682 -0500
timewarp(dev):018> now.class
=> Time
timewarp(dev):019> current = Time.current
=> 2025-12-07 17:18:50.066319000 PST -08:00
timewarp(dev):020> current.class
=> ActiveSupport::TimeWithZone
timewarp(dev):021> current.zone
=> "PST"
timewarp(dev):022> now.zone
=> "EST"

\end{minted}
\end{frame}


\begin{frame}[containsverbatim]
\frametitle{Time.zone}

\mintinline{ruby}{Time.zone}, by default, will use the \mintinline{ruby}{TimeZone} set in the application config.\footnote{See \href{https://api.rubyonrails.org/v8.1/classes/ActiveSupport/TimeZone.html}{ \mintinline{ruby}{TimeZone}  Documentation}. Also look at the \href{https://api.rubyonrails.org/v8.1/classes/Time.html\#method-c-zone}{definition of the \mintinline{ruby}{zone} method in the Rails \mintinline{ruby}{Time} documentation}. It is also possible to set it as something else on a per request basis.}

\end{frame}

\begin{frame}[containsverbatim]
\frametitle{Time.current and Time.zone.now}


\mintinline{ruby}{Time.current} is generally the same as \mintinline{ruby}{Time.zone.now}.

\mintinline{ruby}{Time.current} is slightly superior, in that it also handles a scenario somehow \mintinline{ruby}{Time.zone} is  \mintinline{ruby}{nil}\footnote{See \href{https://github.com/rails/rails/blob/v8.1.1/activesupport/lib/active_support/core_ext/time/calculations.rb\#L39}{definition of  \mintinline{ruby}{current}}}

\begin{minted}[fontsize=\scriptsize,bgcolor=OneDarkBG, bgcolorpadding=1em,breaklines]{ruby}
timewarp(dev):001> current = Time.current
=> 2025-12-07 06:26:14.626656000 PST -08:00
timewarp(dev):002> time_in_app_zone = Time.zone.now
=> 2025-12-07 06:26:17.387557000 PST -08:00
\end{minted}

\end{frame}


\begin{frame}[containsverbatim]
\frametitle{TimeWithZone and TimeZones}

Let's see the times in some specified time zones.\footnote{See \href{https://api.rubyonrails.org/v8.1/classes/ActiveSupport/TimeZone.html}{\mintinline{ruby}{TimeZone} docs} for options}
\begin{minted}[fontsize=\scriptsize,bgcolor=OneDarkBG, bgcolorpadding=1em,breaklines]{ruby}
timewarp(dev):001> current = Time.current
=> 2025-12-07 06:26:14.626656000 PST -08:00
timewarp(dev):002> time_in_app_zone = Time.zone.now
=> 2025-12-07 06:26:17.387557000 PST -08:00
timewarp(dev):003> now_in_central = Time.now.in_time_zone("Central Time (US & Canada)")
=> 2025-12-07 08:26:37.305944000 CST -06:00
timewarp(dev):004> now_in_central.zone
=> "CST"
timewarp(dev):005> now_in_tehran = Time.now.in_time_zone("Tehran")
=> 2025-12-07 17:57:08.617554000 +0330 +03:30
timewarp(dev):006> now_in_tehran.zone
=> "+0330"
\end{minted}
(I made the examples on Sunday)
\end{frame}

\begin{frame}[standout]
  \centering\large
  How many possible times/timezones for any one time in any one app?
\end{frame}

\begin{frame}[standout]
  \centering\Huge 4
  
  \large
  (At least 4)
  
It can be helpful to think of there being 4 main time zones\footnote{I first saw these 4 in this \href{https://thoughtbot.com/blog/its-about-time-zones}{thoughtbot article} but I've also looked into Rails docs to find info which backs this all up}
\end{frame}

\begin{frame}[containsverbatim]
\frametitle{4 Kinds of Times}


\begin{enumerate} 

\item system time
\item application time
\item database time
\item the user's time
\end{enumerate} 
\end{frame}

\begin{frame}[containsverbatim]
\frametitle{4 Kinds of Times}
\framesubtitle{We've already seen the first 2}

\begin{enumerate} 

\item system time
\begin{itemize}
\item My computer timezone is EST
\item An instance of "regular" \mintinline{ruby}{Time} had EST  \mintinline{ruby}{zone}
\end{itemize}
\item application time
\begin{itemize}
\item We configured our application timezone to be Pacific Time
\item  An instance of \mintinline{ruby}{TimeWithZone} (without zone specified) had PST  \mintinline{ruby}{zone}
\end{itemize}

\end{enumerate} 
\end{frame}

\begin{frame}[containsverbatim]
\frametitle{More on System time}


System time will be the time on the server/ computer/ machine where the app is running. (Can vary!)

Not sure that Rails uses the term, system time, but I think it's what rails will refer to as:
\begin{itemize}
\item \mintinline{ruby}{"ENV['TZ']"}
\item  "Ruby's *process* timezone" \footnote{I can talk more about this if there's time, but I say this based info \href{https://api.rubyonrails.org/v8.1/classes/ActiveSupport/TimeWithZone.html}{here} and \href{https://github.com/rails/rails/blob/v8.1.1/activesupport/lib/active_support/core_ext/date/conversions.rb\#L81}{here}}
\item  \mintinline{ruby}{:local} (not to be confused with the method \mintinline{ruby}{local} on \mintinline{ruby}{TimeZone})\footnote{\href{https://api.rubyonrails.org/v8.1/classes/ActiveSupport/TimeZone.html\#method-i-local}{Docs on method \mintinline{ruby}{local}}}

\end{itemize}
\end{frame}

\begin{frame}[containsverbatim]
\frametitle{More on database time}


Database time by default is in UTC time.\footnote{\href{https://guides.rubyonrails.org/v8.1/configuring.html\#config-active-record-default-timezone}{Rails config guide}}

I lean towards leaving it this way.\footnote{\href{https://stackoverflow.com/questions/6118779/how-to-change-default-timezone-for-active-record-in-rails}{See discussion here for issues with changing it}}

\end{frame}

\begin{frame}[containsverbatim]
\frametitle{More on user time}


You can set specific times for your users.\footnote{See Rails example of how to set \href{https://api.rubyonrails.org/v8.1/classes/Time.html\#method-c-zone-3D}{here}}

\end{frame}

\begin{frame}[standout]
  \centering
  
  \large
Enough time chaos!

What should we do?
\end{frame}

  \begin{frame}
\frametitle{Recommendations}

  
 \large 
Use \mintinline{ruby}{Time} (or \mintinline{ruby}{TimeWithZone}) instead of \mintinline{ruby}{DateTime}
  \vfill
\normalsize
There's a \mintinline{ruby}{rubocop DateTime} cop to help enforce this\footnote{See Rubocop Docs about \href{https://docs.rubocop.org/rubocop/1.81/cops_style.html\#styledatetime}{\mintinline{ruby}{DateTime} cop}}  

 \end{frame}
 
 
   \begin{frame}
\frametitle{Recommendations}

 \large 
 Use \mintinline{ruby}{ActiveSupport::TimeWithZone}) over just generic Ruby \mintinline{ruby}{Time}. 
 
   \vfill
 \normalsize
There is a \mintinline{ruby}{rubocop-rails TimeZone} cop that can be enabled to enforce this.\footnote{See Rubocop Rails Docs about  \href{https://docs.rubocop.org/rubocop-rails/2.34/cops_rails.html\#railstimezone}{\mintinline{ruby}{TimeZone} cop}}  

 \end{frame}
 
    \begin{frame}
\frametitle{Recommendations}

 \large 
 Use \mintinline{ruby}{datetime} column type for your migration when you want a date and a time.

 
   \vfill
 \normalsize
Note: The default settings for timestamps in Rails adds them  \mintinline{ruby}{datetime}s in the schema and are \mintinline{ruby}{TimeWithZone} values.\footnote{See schema examples above, and \href{https://api.rubyonrails.org/v8.1/classes/ActiveRecord/Timestamp.html}{Timestamp Docs}}

 \end{frame}
 
\begin{frame}
\frametitle{Recommendations}

 \large 
Remember that you stand on the shoulders of giants who have also struggled with time and time zones!

 \end{frame}

\End
\begin{frame}[plain,standout]
  \centering
  
  Thank you!!
  Questions?
  \vfill

  \footnotesize These Beamer slides made with a customized version of the theme \href{https://github.com/piazzai/arguelles}{Arguelles} 

\end{frame}

  
\end{document}
